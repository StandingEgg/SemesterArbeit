\documentclass{article}
\title{Semester Thesis}
\author{Yifei Wang}
\usepackage{amsmath}
\usepackage{multirow}
\usepackage{graphicx}
\begin{document}
\maketitle

\section{Acknowledgements}
First of all, I would like to express my sincere gratitude to my supervisor Johannes Rutzmoser for the continuous support of my study and related research, for his patience, motivation, and immense knowledge I am deeply grateful of his help in the completion of this thesis.  His guidance helped me in all the time of finite element research and get in with Finite- Element- Research- code. I could not have imagined having a better advisor and mentor for my study in Finite- Element- Method. 

\section{Motivation}
AMfe Toolbox is developed for finite element method (FEM) research. The aim for this develop is to solve and analyse FEM problems. The functions in AMfe still need to be extend. The function for calculation of nodal displacements is completed in AMfe Toolbox. Stress and strain calculation are still empty. Stress and strain calculations are of interest because in structural analysis and design the stresses are often very important to the engineer. For that reason, functions for solving of stress and strain will be added in AMfe Toolbox. When stress is able to export, a research to increase the accuracy of results is worthy of consideration. To improve the accuracy of the results, stress recovery is a approach to extrapolate the element solution to nodal solution. The goal is to get as much accuracy from the computed displacements while keeping the computational effort reasonable. When the most important functions are done and works well, it is time to check if AMfe Toolbox export the reasonable results. The results of stress and strain from AMfe Toolbox will be compared with a commercial FEM software - ANSYS. The compare of each type of element will be recorded.

\section{Structure of Study}

 \begin{figure}
 	\begin{center}
 		\includegraphics[width=12cm,clip]{Overview.pdf}			
 		\caption{Flow chart of structure} \label{fig: Overview}
 	\end{center}
 \end{figure}



This study can be divided into three parts.  Figure \ref{fig: Overview} shows the flow chart of structure in this study. The first part is marked with green colour. 




In compare to the commercial Finite- Element- Software is 

This analysis step is sometimes called postprocessing because it happens after the main processing step — the calculation of nodal displacements — is completed. 
\section{Introduction} 
In the stiffness method of solution solved in existing AMfe-Toolbox, the stresses are calculated from the computed displacements, and are thus derived quantities. The accuracy of derived quantities is generally lower than that of primary quantities (the displacements). The accuracy level of displacements is usually more than tenfold higher than that of stresses. At boundaries the stresses are even more inaccurate than that.
For that reason it is necessary to develop techniques that improve the accuracy of the computed stresses. The aim is to enhance the accuracy from the computed displacements while keeping the computational effort reasonable. This technique is called stress recovery in the finite element method.
\section{Calculation of Element Strains and Stresses } In elastic materials, stresses $\sigma$ are directly related to strains $\varepsilon$ at each point through the elastic constitutive equations 
\begin{equation}
	\sigma = E\varepsilon 
\end{equation}
It follows that the stress computation procedure begins with strain computations, and that the accuracy of stresses depends on that of strains. Strains, however, are seldom saved or printed. 
In the following we focus our attention on two-dimensional isoparametric elements, as the computation of strains, stresses and axial forces in bar elements is strightforward. 
Suppose that we have solved the master stiffness equations
 
\begin{equation}
	Ku = f
\end{equation}
for the node displacements $u$. To calculate strains and stresses we perform a loop over all defined elements. Let $\varepsilon$ be the element index of a specific two-dimensional isoparametric element encoun- tered during this loop, and $u(\varepsilon)$ the vector of computed element node displacements. The strains at any point in the element can be related to these displacements as 
\begin{equation}
	\varepsilon = Bu(\varepsilon) 
\end{equation}
where $B$ is the strain-displacement matrix assembled with the x and y derivatives of the element shape functions evaluated at the point where we are calculating strains. The corresponding stresses are given by 
\begin{equation}
	\sigma = E\varepsilon = EBu 
\end{equation}
In the applications it is of interest to evaluate and report these stresses at the element nodal points located on the corners and possibly midpoints of the element. These are called element nodal point stresses.
It is important to realize that the stresses computed at the same nodal point from adjacent elements will not generally be the same, since stresses are not required to be continuous in displacement- assumed finite elements. This suggests some form of stress averaging can be used to improve the stress accuracy, and indeed this is part of the stress recovery technique further. The results from this averaging procedure are called nodal point stresses.


\section{Gauss integration}
Numerical integration plays a role in finite element. Gauss integration, as a very efficient approach compared to other numerical integration, integrate a function $f(\xi)$ on the spatial domain $\Omega$ is replaced by a summation of certain function values at the so-called Gauss points $\tilde{\xi}$ which are each multiplied by a scalar (weight) $\omega$.\\

In total $numgp_1$ $\times$ $numggp_2$ $\times$ $numgp_3$ Gauss points are used: \\

$\iiint \limits_E f\left(\xi\right) \mathrm{d\xi_1}\mathrm{d\xi_2}\mathrm{d\xi_3}$ = $\displaystyle\sum_{i=1}^{numgp_1}\displaystyle\sum_{j=1}^{numgp_2} \displaystyle\sum_{k=1}^{numgp_3} f\left(\tilde{\xi_1^i}\tilde{\xi_2^j}\tilde{\xi_3^k}\right)\cdot w_1^i w_2^j w_3^k$ \newline

In which we used corresponding weights for each of the three spatial directions. This allows to use Gauss points coordinates and weights tabulated for the one-dimensional case. \\

The location and weights of these points can be given analytically in Table \ref{tab: Guass table}.

\begin{table}
	\begin{center}
		\caption{Gauss-Legendre points and weights}\label{tab: Guass table}
		\begin{tabular}{cccc}
			$numgp$         & i & $\tilde{\xi^i}$ & $\omega^i$  \\ \hline
			1                           &1& 0 & 2   \\ \hline
			\multirow{2}{*}{2} &1& $-1/\sqrt{3}$ & 1   \\
			&2& $+1/\sqrt{3}$ & 1   \\ \hline
			\multirow{3}{*}{3} &1& $-\sqrt{3/5}$ &5/9 \\
			&2& 0                &8/9 \\
			&3& $+\sqrt{3/5}$ &5/9 \\ \hline
			\multirow{4}{*}{4}&1& $-\sqrt{\left(15+\sqrt{120}\right)/35}$ & $\left(18-\sqrt{30}\right/36)$ \\
			&2& $-\sqrt{\left(15-\sqrt{120}\right)/35}$ & $\left(18+\sqrt{30}\right/36)$ \\
			&3& $+\sqrt{\left(15-\sqrt{120}\right)/35}$ & $\left(18+\sqrt{30}\right/36)$ \\
			&4& $+\sqrt{\left(15+\sqrt{120}\right)/35}$ & $\left(18-\sqrt{30}\right/36)$ \\ \hline
			\multirow{5}{*}{5}&1& $-1/3\sqrt{5+2\sqrt{10/7}}$ & $\left(332-13\sqrt{70}\right)/900$ \\
			&2& $-1/3\sqrt{5-2\sqrt{10/7}}$ & $\left(332+13\sqrt{70}\right)/900$ \\
			&3& 0 & 128/255 \\
			&4& $+1/3\sqrt{5-2\sqrt{10/7}}$ & $\left(332+13\sqrt{70}\right)/900$ \\
			&5& $+1/3\sqrt{5+2\sqrt{10/7}}$ & $\left(332-13\sqrt{70}\right)/900$ \\ \hline
		\end{tabular}
	\end{center}
	
\end{table}

\section{Shape functions}
\subsection{Quadrilateral Elements}
The standard approach for the definition of shape functions is to chose them as simple polynomials which are associated to nodes. One element includes $n$ nodes and $n$ shape functions, where the $i$-th shape function takes the value 1 at the $i$-th node in reference coordinates, then the $j$-th shape function can be written like this: \\

\begin{center}
	if $j$ = $i$ then $N_j\left(\hat{\xi_i}\right) = 1$ \\
	if $j$ $\neq$ $i$ then $N_j\left(\hat{\xi_i}\right) = 0$
\end{center}

for $i$,$j$ = 1,...,n. \\
In Figure \ref{fig: shape_func}, linear, quadratic shape function are shown together with corresponding node positions. When it comes to a element with three nodes, as shown in the right of Figure \ref{fig: shape_func}. For each shape functions, there are three restrictions:

\begin{figure}
	\begin{center}
			\includegraphics[width=10cm,clip]{shape_func.pdf} \label{fig: shape_func}			
			\caption{Linear shape function and quadratic shape function.}
	\end{center}
\end{figure}

\begin{center}
	$N_1\left(\hat{\xi_1} = -1\right) = 1$, $N_1\left(\hat{\xi_2} = 0\right) = 0$, $N_1\left(\hat{\xi_3} = 1\right) = 0$;\\
	$N_2\left(\hat{\xi_1} = -1\right) = 0$, $N_2\left(\hat{\xi_2} = 0\right) = 1$, $N_2\left(\hat{\xi_3} = 1\right) = 0$;\\
	$N_3\left(\hat{\xi_1} = -1\right) = 0$, $N_3\left(\hat{\xi_2} = 0\right) = 0$, $N_3\left(\hat{\xi_3} = 1\right) = 1$
\end{center}

In general quadratic polynomial, the shape functions have three coefficients, and it can be constructed as follows:
\begin{center}
	 $p_{quad}\left(\xi\right) = c_0 + c_1\xi + c_2\xi^2$
\end{center}

Lagrange polynomials are a general class of polynomials with the node association property: a Lagrange polynomial $l_k^n-1$ (in one coordinate $\xi$) of order $n-1$ passes through $n$ nodes with coordinates $\bar{\xi^j}\left(j = 1,...,n\right)$ of which the single node $k$ evaluates unity $\left(l_k^{n-1} \left(\bar{\xi^k}\right)=1\right)$ and every other node results in zero $\left(l_k^{n-1} \left(\bar{\xi^k}\right)=0  \quad \text{for all} \quad j \neq k\right)$ .

\begin{center}
	$l_k^{n-1} \left( \xi \right) = \prod_{j = 1, j \neq k}^{n} \frac{\xi - \bar{\xi^1}}{\bar{\xi^k - \bar{\xi^j}}} = \frac{\left(\xi - \bar{\xi}^1\right)\cdot\cdot\cdot \left( \xi - \bar{\xi}^{k-1}\right) \left(\xi - \bar{\xi}^{k+1}\right)\cdot\cdot\cdot \left(\xi -\bar{\xi}^n\right)}{\left(\bar{\xi}^k - \bar{\xi}^1\right)\cdot\cdot\cdot \left( \bar{\xi}^k - \bar{\xi}^{k-1}\right) \left(\bar{\xi}^k - \bar{\xi}^{k+1}\right)\cdot\cdot\cdot \left(\bar{\xi}^k -\bar{\xi}^n\right)}$ \\[4mm] \quad $k = 1,2....,n$
\end{center}
 For instance linear, i.e. order 1, Lagrange polynomials are achieved with $n-1=1$ $\Rightarrow$ $n=2$ nodes. Using $\bar{\xi}^1 = -1$ and $\bar{\xi}^2 = +1$ results in: $l_1^1\left(\xi\right) = -1/2\left(\xi - 1\right)$ and $l_2^1\left(\xi\right) = 1/2\left(\xi + 1\right)$.
 Then the linear shape functions can be simply constructed as $N_1 =l_1^1 = 1/2\left(1-\xi\right)$ and $N_2 = l_2^1 = 1/2\left(1+\xi\right)$.
 
 \subsection{Triangular and Tetrahedral Elements}
 Due to the higher flexibility of triangles and tetrahedra for meshing complex geometries, they are often preferred over quadrilaterals or hexahedral. The shape functions are given in an analog way but are expressed in area and volume coordinates, respectively. Figure \ref{tri&tet} shows the geometries and shape functions. These coordinates represent area and volume fractions, a visualization is given below:
 \begin{center}
 	$L_1 = \frac{\text{area}P23}{\text{area}123}$, \quad $V_1 = \frac{\text{area}P234}{\text{area}1234}$ \\[4mm]
 	$\sum L_i = 1$ \quad\quad $\sum V_i = 1$
 \end{center}
 
 \begin{figure}
 	\begin{center}
 		\includegraphics[width=11cm,clip]{Tri&Tet.pdf}			
 		\caption{triangular and tetradral elemnets.} \label{fig: tri&tet}
 	\end{center}
 \end{figure}
 
 
 

\section{Extrapolation from Gauss Points}
\subsection{Quad4}
This will be explained for the four-node bilinear quadrilateral. The normal Gauss integration
rule for element stiffness evaluation is 2 $\times$ 2, as illustrated in Figure \ref{fig: Quad4_1}.
The stresses are calculated at the Gauss points, which are identified as $k_1^G$, $k_2^G$, $k_3^G$ and $k_4^G$ in Figure \ref{fig: Quad4_1}. Point $k_i^G$ is closest to node $k_i^E$ so it is seen that Gauss point numbering essentially follows element node numbering in the counterclockwise sense. The natural coordinates of these points are listed in Table \ref{tab: Quad4}. The stresses are evaluated at these Gauss points by passing these natural coordinates to the shape function subroutine. Then each stress component is “carried” to the corner nodes $k_1^E$ through $k_4^E$ through a bilinear extrapolation based on the computed values at $k_1^G$ through $k_4^E$.
To understand the extrapolation procedure more clearly it is convenient to consider the region bounded by the Gauss points as an “internal element” or “Gauss element”. This interpretation is depicted in Figure \ref{fig: Quad4_1}. The Gauss element, denoted by (G), is also a four-node quadrilateral. The coordinate of node for element and Gauss element can be represented as $k_i^G$ and $k_i^E$, respectively. Its quadrilateral (natural) coordinates are denoted by $\xi′$ and $\eta′$. 
These are linked to $\xi′$ and $\eta′$ by the simple relations from Gauss-Legendre quadrature in Table \ref{tab: Quad4}.
\begin{equation}
k_i^G = \frac{k_i^E}{\sqrt{3}},\quad
k_i^E= k_i^G\sqrt{3}
\end{equation}




\begin{figure}[h]
	\begin{center}
		\includegraphics[width=8cm,clip]{Quad4_1.pdf}			
		\caption{Quad4 in element coordinate and Gauss element coordinate.}	\label{fig: Quad4_1}
	\end{center} 
\end{figure}

\begin{figure}[h]
	\begin{center}
		\includegraphics[width=8cm,clip]{Quad4_2.pdf}			
		\caption{Equation of side opposite corner 1 for Quad4.} \label{fig: Quad4_2}
	\end{center} 
\end{figure}

\begin{table}
	\centering
	\caption{Natural Coordinate of Quad4}
	\label{tab: Quad4}
	\begin{tabular}{p{1cm}ccccp{1cm}cccc}			
		\hline
		Corner node\centering& $\xi$& $\eta$& $\xi'$& $\eta'$& Gauss node\centering& $\xi$& $\eta$& $\xi'$& $\eta'$ \\
		\hline
		1\centering& -1& -1& $-\sqrt{3}$& $-\sqrt{3}$& 1'\centering& 1/$\sqrt{3}$& -1/$\sqrt{3}$& -1& -1 \\
		2\centering& +1& -1& $+\sqrt{3}$& $-\sqrt{3}$& 2'\centering& 1/$\sqrt{3}$& 1/$\sqrt{3}$& +1& -1 \\
		3\centering& +1& +1& $+\sqrt{3}$& $+\sqrt{3}$& 3'\centering& 1/$\sqrt{3}$& 1/$\sqrt{3}$& +1& +1\\
		4\centering& -1& +1& $-\sqrt{3}$& $+\sqrt{3}$& 4'\centering& -1/$\sqrt{3}$& 1/$\sqrt{3}$& -1& +1\\
		\hline
	\end{tabular}
\end{table}		


The element geometry and natural coordinates are shown in Figure \ref{fig: Quad4_1}. Only one type of node (corner)  and associated shape function is present. Consider node 1 as typical. Inspection of Figure \ref{fig: Quad4_2} suggests trying

\begin{equation} \label{eq: Quad4_1}
N_1^e = c_1L_{2-3}L_{3-4}
\end{equation}
This plainly vanishes over nodes 2, 3 and 4, and can be normalized to unity at node 1 by adjusting $c_1$. By construction it vanishes over the sides 2-3 and 3-4 that do not belong to 1. The equation of side 2-3 is $\xi = 1$, or $\xi - 1 = 0$. The equation of side 3-4 is $\eta = 1$, or $\eta - 1 = 0$. Replacing in Equation (\ref{eq: Quad4_1}) yields

\begin{equation}
N_1^e\left(\xi, \eta\right) = c_1 \left( \xi -1 \right) \left( \eta - 1\right) = c_1 \left(1 - \xi\right) \left( 1 - \eta \right)
\end{equation}
To find $c_1$, evaluate at node 1, the natural coordinates of which are $\xi = \eta = -1$:

\begin{equation}
N_1^e \left(-1, -1 \right) = c_1 \times 2 \times 2 = 4c_1 = 1
\end{equation}

Hence $c_1 = \frac{1}{4}$ and the shape functions is

\begin{equation}
N_1^e = \frac{1}{4} \left(1 - \xi\right) \left( 1 - \eta\right)
\end{equation}

For the other three nodes the procedure is the same, traversing the element cyclically. It can be verified that the general expression of the shape functions for this element is 

\begin{equation}
N_i^e = \frac{1}{4} \left( 1 + \xi_i \xi\right) \left(1 + \left(\eta_i \eta\right)\right)
\end{equation}

Following this general expression, the shape functions of Node 2, 3 and 4 are demonstrated as

\begin{equation}
N_2^e = \frac{1}{4} \left(1 + \xi\right) \left( 1 - \eta\right)
\end{equation}

\begin{equation}
N_3^e = \frac{1}{4} \left(1 + \xi\right) \left( 1 + \eta\right)
\end{equation}

\begin{equation}
N_4^e = \frac{1}{4} \left(1 - \xi\right) \left( 1 + \eta\right)
\end{equation}


\subsection{Quadrangle with quadratic shape function and  eight nodes: Quad8}

The Nine-Node biquadratic quadrilateral has three types of shape functions, which are associated with corner nodes, midside nodes and center node, respectively. The element coordinate and Gauss element coordinate are illustrated in Figure \ref{fig: Quad8_1} \\
The lines whose product is used to construct three types of shape functions are illustrated in Figure  for nodes 1, 5 and 9, respectively. The technique has been sufficiently illustrated in previous examples. Here the summary of calculation for nodes 1, 5 and 9, which are taken as representatives of the three types: The three types of shape function are represented in Figure \ref{fig: Quad8_2}.

\begin{equation} \label{eq: Quad8_1}
N_1^e = c_1 L_{2-3}L_{3-4}L_{5-7} L_{6-8} = c_1 \left(\xi - 1\right) \left(\eta -1\right) \xi \eta
\end{equation}

\begin{equation} \label{eq: Quad8_2}
N_5^e = c_5L_{2-3} L_{1-4} L_{6-8} L_{3-4} = c_5 \left(\xi -1 \right) \left( \xi +1\right) \eta \left( \eta -1\right) = c_5 \left(1-\xi^2\right) \eta \left(1 - \eta\right)
\end{equation}

\begin{equation} \label{eq: Quad8_3}
N_9^e = c_9L_{1-2} L_{2-3} L_{3-4} L_{4-1} = c_9 \left(\xi -1 \right) \left( \eta - 1\right) \left(\xi + 1\right) \left( \eta + 1\right) = c_9 \left(1-\xi^2\right) \left(1 - \eta^2\right)
\end{equation}

Imposing the normalization conditions the results are 

\begin{center}
$c_1 = \frac{1}{4}$, \quad $c_5 = -\frac{1}{2}$, $c_9 = 1$
\end{center}

By following this approach all the shape functions can be calculated

\begin{equation}
N_1 = \frac{1}{4} \left(\xi - 1\right) \left( \eta -1 \right) \cdot\xi \cdot\eta
\end{equation}

\begin{equation}
N_2 = \frac{1}{4} \left(\xi + 1\right) \left( \eta -1 \right) \cdot\xi \cdot\eta
\end{equation}

\begin{equation}
N_3 = \frac{1}{4} \left(\xi + 1\right) \left( \eta + 1 \right) \cdot\xi \cdot\eta
\end{equation}

\begin{equation}
N_4 = \frac{1}{4} \left(\xi - 1\right) \left( \eta + 1 \right) \cdot\xi \cdot\eta
\end{equation}

\begin{equation}
N_5 = \frac{1}{2} \left(1 + \xi \right) \left( 1 + \xi \right)  \left( \eta - 1\right) \cdot\eta
\end{equation}

\begin{equation}
N_6 = \frac{1}{2} \left(1 + \eta \right) \left( 1 - \eta \right)  \left( \xi + 1\right) \cdot\xi
\end{equation}

\begin{equation}
N_7 = \frac{1}{2} \left(1 + \xi \right) \left( 1 + \xi \right)  \left( \eta + 1\right) \cdot\eta
\end{equation}

\begin{equation}
N_8 = \frac{1}{2} \left(1 + \eta \right) \left( 1 - \eta \right)  \left( \xi - 1\right) \cdot\xi
\end{equation}

\begin{equation}
N_9 = \left(1 + \xi \right) \left( 1 - \xi \right)  \left( 1 + \eta \right) \left(1 - \eta \right)
\end{equation}



\begin{figure}[h]
	\begin{center}
		\includegraphics[width=10cm,clip]{Quad8_1.pdf}			
		\caption{Quad8 in element coordinate and Gauss element coordinate.} \label{fig: Quad8_1}
	\end{center} 
\end{figure}

\begin{figure}[h]
	\begin{center}
		\includegraphics[width=6cm,clip]{Quad8_2.pdf} \label{fig: Quad8_2}	
		\includegraphics[width=6cm,clip]{Quad8_3.pdf}	
		\includegraphics[width=6cm,clip]{Quad8_4.pdf}
		\caption{Equation of side opposite corner 1 for Quad8.}
	\end{center} 
\end{figure}

\subsection{Triangle with three nodes: Tri3}
The geometry of the 3-node triangle shown in Figure \ref{fig: Tri3_1} is specified by the location of its three corner nodes on the $\left\{x, y\right\}$ plane \\

The shape function for triangular element has a different form, which compares with quadrilateral elements. The three shape functions have the simply own coordinates - the triangular coordinates: $N_i = \xi_i$ for $i = 1, 2, 3.$ The shape function can be derived from a method as follows: \\

The equation of the triangle side opposite to node $i$ is $L_{j-k} = \xi_i = 0$, where $j$ and $k$ are the cyclic permutations of $i$. Here symbol $L_{j-k}$ denotes the left hand side of the homogeneous equation of the natural coordinate line that passes through node points $j$ and $k$. See Figure \ref{fig: Tri3_2} for $i = 1$, $j=2$ and $k = 3$. Hence the obvious suppose is:
\begin{equation}
N_i^e = c_iL_i
\end{equation}
To find $c_1$, evaluate $N_1^e\left(\xi_1, \xi_2, \xi_3\right)$ at node 1. The triangular coordinates of this node are $\xi_1$ =1, $\xi_2 = \xi_3 = 0$.  $N_1^e\left(1,0,0\right) = c_1 \times 1 = 1$. so $c_1 = 1$ and analogically $c_2 = 1$, $c_3 = 1$

\begin{equation}
N_1^e = L_1 , \quad N_2^e = L_2, \quad N_3^e = L_3
\end{equation}


\begin{figure}[h]
	\begin{center}
		\includegraphics[width=8cm,clip]{Tri3_1.pdf}			
		\caption{Tri3 in element coordinate and Gauss element coordinate.} \label{fig: Tri3_1}
	\end{center} 
\end{figure}

\begin{table}
	\centering
	\caption{Natural Coordinate of Tri3} \label{tab: Tri3}
	\begin{tabular}{p{1cm}ccccp{1cm}cccc}	
		
		\hline
		Corner node\centering& $\xi$& $\eta$& $\xi'$& $\eta'$& Gauss node\centering& $\xi$& $\eta$& $\xi'$& $\eta'$ \\
		\hline
		1\centering& -1& -1& -5/3& -5/3& 1'\centering& -2/3& -2/3& -1& -1 \\
		2\centering& +1& -1& +7/3& -5/3& 2'\centering&+1/3 & -2/3& +1& -1 \\
		3\centering& -1& +1& -5/3& +7/3& 3'\centering& -2/3& +1/3& -1& +1\\
		\hline
		
	\end{tabular}
\end{table}			



\begin{figure}[h]
	\begin{center}
		\includegraphics[width=6cm,clip]{Tri3_2.pdf}			
		\caption{Equation of side opposite corner 1 for Tri3.} \label{fig: Tri3_2}
	\end{center} 
\end{figure}

\subsection{Second order triangle with six nodes: Tri6}
The geometry of the six-node quadratic is shown in Figure XXX. Inspection reveals two types of nodes: corners(1, 2 and 3) and midside nodes(4, 5 and 6). For both cases it is necessary to product the two linear functions in the triangular coordinates because the shape function should be quadratic. These functions are illustrated in Figures XXX for corner node 1 and midside node 4, respectively. \\
For corner node 1, inspection of Figure \ref{fig: Tri6_2} at top left side suggests trying
\begin{equation}
N_1^e = c_1L_{2-3}L_{4-6}
\end{equation}
$N_1^e$ will vanish over 2-5-3 and 4-6. This makes the function zero at node 2,3,4,5,6, as is obvious upon inspection of Figure XXX, while being nonzero at 1. The value can be adjusted to be unity if $c_1$ is appropriately chosen. The equations of the lines that appear (XXX equation upper) are
\begin{equation}
L_{2-3}: \xi_1 = 0, \quad L_{4-6}: \xi_1 - \frac{1}{2} = 0
\end{equation}
Replacing into (eq upper upper)

\begin{equation}
N_1^e = c_1\xi_1 \left(\xi_1 - \frac{1}{2}\right)
\end{equation}
Same as triangle with three nodes, $N_1^e\left(1,0,0\right) = c_1 \times 1 \times \frac{1}{2} = 1$. Then $c_1 = 2$ can be calculated and finally 

\begin{equation}
N_1^e = 2\xi_1\left(\xi_1 - \frac{1}{2}\right) = \xi_1 \left(2 \xi_1 -1 \right)
\end{equation} 

For midside node 4, inspection of Figure XXX suggests trying

\begin{equation}
N_4^e = c_4L_{2-3}L_{1-3}
\end{equation}
The equation of sides $L_{2-3}$ and $L_{1-3}$ are $\xi_1 = 0$ and $\xi_2 = 0$, respectively. Therefore $N_4^e\left(\xi_1, \xi_2, \xi_3\right) = c_4\xi_1\xi_2$. To find $c_4$, evaluate this function at node 4, the triangular coordinates of which are $\xi_1 = \xi_2 = \frac{1}{2}$,$\xi_3 = 0$. Then $N_4^e\left(\frac{1}{2}, \frac{1}{2}, 0\right) = c_4 \times \frac{1}{2} \times \frac{1}{2} = 1$. Hence $c_4 = 4$, which the shape function gives

\begin{equation}
N_4^e = 4\xi_1\xi_2
\end{equation}
The rest shape function can be calculated by using same approach. 
\begin{center}
	
\end{center}

\begin{figure}[h]
	\begin{center}
		\includegraphics[width=8cm,clip]{Tri6_1.pdf} 		
		\caption{Tri6 in element coordinate and Gauss element coordinate.} \label{fig: Tri6_1}	
	\end{center} 
\end{figure}

\begin{figure}[h]
	\begin{center}
		\includegraphics[width=6cm,clip]{Tri6_2.pdf}		
		\includegraphics[width=6cm,clip]{Tri6_3.pdf}		
		\caption{Equation of side opposite corner 1 for Tri6.} \label{fig: Tri6_2}
		
	\end{center} 
\end{figure}


\subsection{Tetraeder element with four nodes: Tet4}

\begin{figure}[h]
	\begin{center}
		\includegraphics[width=8cm,clip]{Tet4_1.pdf}			
		\caption{Tri4 in element coordinate and Gauss element coordinate.}
	\end{center} 
\end{figure}

\begin{table}
	\centering
	\caption{Tetrahedral Coordinate of Tet4}
	\begin{tabular}{ccccccccc}			
		\hline
		Corner node\centering& L1& L2& L3& L4& L1'& L2'& L3'& L4'\\ \hline
		1\centering& 1& 0& 0& 0& $\alpha$& $\beta$& $\beta$& $\beta$\\
		2\centering& 0& 1& 0& 0& $\beta$& $\alpha$& $\beta$& $\beta$\\
		3\centering& 0& 0& 1& 0& $\beta$& $\beta$& $\alpha$& $\beta$\\
		4\centering& 0& 0& 0& 1& $\beta$& $\beta$& $\beta$& $\alpha$\\
		\hline
		Gauss node\centering& L1& L2& L3& L4& L1'& L2'& L3'& L4' \\ \hline
		1'\centering& $\alpha$& $\beta$& $\beta$& $\beta$& 1& 0& 0& 0  \\
		2'\centering&$\beta$ & $\alpha$& $\beta$& $\beta$& 0& 1& 0& 0 \\
		3'\centering& $\beta$& $\beta$& $\alpha$& $\beta$& 0& 0& 0& 1\\
		4'\centering& $\beta$& $\beta$& $\beta$& $\alpha$& 0& 0& 0& 1\\
		\hline
		$\alpha$ = 0.58541020; $\beta$ = 0.13819660&&&&&&&&\\
		\hline 		    
	\end{tabular}	
\end{table}		



\section{Coordinate Transformations}
Quantities that are closely linked with the element geometry are best expressed in triangular co- ordinates. On the other hand, quantities such as displacements, strains and stresses are usually expressed in the Cartesian system. Thus we need transformation equations through which it is possible to pass from one coordinate system to the other.
Cartesian and triangular coordinates are linked by the relation.

\begin{equation}
	$$$\begin{bmatrix}
	1 \\
	x \\
	y \\
	\end{bmatrix}$
	= 
	$\begin{bmatrix}
	1&1&1 \\
	x_1&x_2&x_2 \\
	y_1&y_2&y_2 \\
	\end{bmatrix}$
	$\begin{bmatrix}
	\xi_1 \\
	\xi_2 \\
	\xi_3 \\
	\end{bmatrix}$$$
\end{equation}

\begin{equation}
$$$\begin{bmatrix}
\xi_1 \\
\xi_2 \\
\xi_3 \\
\end{bmatrix}$
= 
$\frac{1}{2A}$$\begin{bmatrix}
x_2y_3-x_3y_2&y_2-y_3&x_3-x_2 \\
x_3y_1-x_1y_3&y_3-y_1&x_1-x_3 \\
x_1y_2-x_2y_1&y_1-y_2&x_2-x_1 \\
\end{bmatrix}$
$\begin{bmatrix}
1\\
x \\
y \\
\end{bmatrix}$
= 
$\frac{1}{2A}$$\begin{bmatrix}
2A_{23}&y_{23}&x_{32} \\
2A_{31}&y_{31}&x_{13} \\
2A_{12}&y_{12}&x_{21} \\
\end{bmatrix}$$
\begin{bmatrix}$$
1 \\
x \\
y \\

\end{bmatrix}$$$
\end{equation}$$$$



\cite[p. 18]{bibid}

\begin{figure}[h]
	\begin{center}
		\includegraphics[width=8cm,clip]{foo.pdf}			
		\caption{Tri4 in element coordinate and Gauss element coordinate.}
	\end{center} 
\end{figure}


\end{document}

